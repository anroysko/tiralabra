\documentclass{article}

\title{Viikkoraportti 1}
\date{15. maaliskuuta 2018}
\author{Antti R\"oysk\"o}

\begin{document}
\maketitle

\noindent
T\"all\"a viikolla p\"a\"atin aiheen, tein github-repon, m\"a\"arittelydokumentin ja t\"am\"an viikkoraportin. Ty\"oh\"on aikaa k\"aytin noin 5 tuntia, aiheen valitsemiseen paljon enemm\"an. Tiesin haluavani tehd\"a jotain joka hy\"odynt\"a\"a Link/Cut treeit\"a, joten p\"a\"atin tehd\"a Dinic'n algoritmin.\\

\noindent
Ohjelman tekemist\"a en ole viel\"a aloittanut. Teen ohjelman C++:lla.\\

\noindent
Gitin k\"aytt\"o on viel\"a minulle aika uutta, mutta onneksi sit\"a on aika helppo oppia.\\

\noindent
Kurssin logistiikka on viel\"a suurilta osin ep\"aselv\"a\"a. Aikataulu n\"aytt\"a\"a silt\"a, ett\"a ohjelma kuuluisi toteuttaa ylh\"a\"alt\"a alas, vaikka t\"ass\"a tapauksessa looginen progressio on alhaalta yl\"os Splay Tree \rightarrow Link/Cut tree \rightarrow Dinic.\\

\noindent
Seuraavalla viikolla selvit\"an miten tarkalleen eri osien on tarkoitus k\"aytt\"a\"a toisiansa, ja koodaan Splay Treen.\\
\end{document}
