\documentclass{article}
\title{Tiralabra-M\"a\"arittelydokumentti}
\author{Antti R\"oysk\"o}
\date{15. Maaliskuuta 2018}
\usepackage[hidelinks]{hyperref}

\begin{document}
\maketitle

\noindent
Ty\"oss\"a toteutetaan Dinic's algorithm maximum flow -ongelman ratkaisemiseen. Se implementoidaan Link/Cut Treen avulla. Toteutetaan My\"os Link/Cut Treen sis\"aisesti k\"aytt\"am\"a Splay Tree. \\

\noindent
Kyseiset tietorakenteet on valittu koska niill\"a saadaan ratkaistua maximum flow -ongelma k\"ayt\"annoss\"a nopeasti, ja asymptioottinen aikavaatimus on vain logaritmisen kertoimen p\"a\"ass\"a parhaasta tunnetusta algoritmist\"a. \\

\noindent
Ohjelma saa sy\"otteeksi suunnatun verkon, edgejen kapasiteetit ja sinkin sek\"a sourcen indeksit verkossa. Ohjelman teen C++:lla. Se tuottaa jonkin maksimaalisen flown kyseiselle graafille.\\

\noindent
Algoritmin aikavaatimus on $O(V\ E\ log(V))$, miss\"a V on vertexien m\"a\"r\"a ja E Edgejen m\"a\"ar\"a.\\

\noindent
\url{https://en.wikipedia.org/wiki/Dinic's_algorithm}\\
\url{https://en.wikipedia.org/wiki/Link/cut_tree}\\
\url{https://en.wikipedia.org/wiki/Maximum_flow_problem}\\
\end{document}
